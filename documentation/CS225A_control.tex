\documentclass[10pt,a4paper,notitlepage]{report}
\usepackage[utf8]{inputenc}
\usepackage{amsmath}
\usepackage{amsfonts}
\usepackage{amssymb}
\begin{document}
\section*{Control Outline}
\paragraph*{}
Our control implementation for simulation is based on giving the robot a desired position
and orientation, and using the known dynamics of the robot to calculate the necessary joint torques
to apply (1). The torques for the translational and rotational parts of the controller
are calculated separately and then combined, with the position and rotation parts denoted
by the subscripts P and R. 
\paragraph*{} 
We use the pseudo-inverse method to 
find the task space mass matrix $\Lambda$
from the joint space matrix (4)(5). We find the task space torques using PD control (6)(7), 
using the error in position (8) for the cartesian force and error in orientation (9) 
for the moments $F_{R}$. 
\paragraph*{}

The error in orientation (9) is found by taking the cross product of each column vector in the 
current rotation matrix $R_{CO}$ by the column vectors in the matrix 
representing the desired rotation $R_{DO}$. This gives a torque axis and magnitude which the proportional rotation gain $k_{pR}$ is applied to.

     

\begin{align}
\tau &= \begin{bmatrix}
\tau_{P}\\
\tau_{R}
\end{bmatrix}+G(q) \\
\tau_{P} &= J^{T}_{P}(\Lambda_{P}F_{P})\\
\tau_{R} &= J^{T}_{R}(\Lambda_{R}F_{R})\\
\Lambda_{P} &= (J_{P}M^{-1}J_{P}^{T})^{-1}\\
\Lambda_{R} &= (J_{R}M^{-1}J_{R}^{T})^{-1} \\
F_{P} &= -k_{pP}*dx - k_{vP}*\dot{x}\\
F_{R} &= -k_{pR}*d\phi - k_{vR}*\omega  \\
dx &= x_{C} - x_{D}\\
d\phi &= -\frac{1}{2}(s_{1C}\times s_{1D}+s_{2C}\times s_{2D}+s_{3C}\times s_{3D})\\
\omega &= J_{R}*\dot{q}\\
\dot{x} &= J_{P}*\dot{q}\\
\end{align}
$R_{CO} = 
\begin{bmatrix}
\vert & \vert & \vert\\
s_{1C} & s_{2C} & s_{3C}\\
\vert & \vert & \vert\\
\end{bmatrix}\\
R_{DO} = 
\begin{bmatrix}
\vert & \vert & \vert\\
s_{1D} & s_{2D} & s_{3D}\\
\vert & \vert & \vert\\
\end{bmatrix}
$
\end{document}